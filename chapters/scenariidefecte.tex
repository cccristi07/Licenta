\chapter{Scenarii pentru defecte și simulări}
\label{chap:detectie}
\section{Definirea defectelor}
Defectele sunt simulate modificând parametrul $C$ din ecuația emitter-ului \eqref{eq:emitter}. Modalitatea prin care se execută în cod simularea unui defect este prin apelarea metodei:

\lstinputlisting[language=Python, caption={Funcție pentru simularea defectelor},label={lst:set_emitter}, firstline=23,lastline=26]{\code/ENWrapper.py}

Parametrii funcției $set\_emitter$ sunt:
\begin{itemize}
\item node\_index - indexul nodului în care se simulează defectul
\item emitter\_val - magnitudinea defectului
\end{itemize}

Metoda mai întâi verifică dacă nodul cu indexul $node_{index}$ reprezintă doar o joncțiune apoi setează magnitudinea defectului în nodul primit cu ajutorul funcției de bibliotecă $\mathbf{ENsetnodevalue}$.

\section{Simulare dinamică pentru defecte în diferite noduri}

În continuare vom considera un scenariu de defect pentru rețea care constă în modificarea succesivă a parametrului de proporționalitate din relația de calcul a debitului de emitter \eqref{eq:emitter}. În imaginile următoare voi considera mai multe magnitudini de defect într-un anumit nod și voi reprezenta grafic răspunsul în timp al rețelei în același nod.

\begin{figure}[H]

\subfloat[Profile cu defect în nodul 14]{%
  \includegraphics[width=0.5\textwidth]{\pics/c3_pics/emitter_node_same/emitter_node14}%
  \label{fig:emitter_node_same14}%
}\qquad
\subfloat[Profil cu defect în nodul 25]{%
  \includegraphics[width=0.5\textwidth]{\pics/c3_pics/emitter_node_same/emitter_node25}%
  \label{fig:emitter_node_same25}\qquad
}

\caption{Rezultate simulări defecte ușoare}
\label{fig:ref_emitter_soft}
\end{figure}

După cum se poate observa în imaginile \ref{fig:ref_emitter_soft} variația emitter-ului într-un nod produce în mod evident o modificarea a modului comun al caracteristicii $timp-presiune$. Din punctul de vedere al magnitudinilor de simulare pentru defecte, am considerat 2 clase de defecte, anume:
\begin{itemize}
\item defecte ușoare (\emph{soft faults}) - cu valorile coeficientului de emitter mai mici de 35
\item defecte puternice (\emph{hard faults}) - cu valorile emitter mai mari de 35 
\end{itemize}

Cele din urmă produc și modificări ale caracteristicii dinamice, introducând distorsiuni sau aplatizări ale mărimilor măsurate. Reprezentarea defectelor hard este reprezentată în figurile de mai jos:

\begin{figure}[H]

\subfloat[Profile cu defect puternic în nodul 14]{%
  \includegraphics[width=0.5\textwidth]{\pics/c3_pics/emitter_node_same/emitter_hard_node14}%
  \label{fig:emitter_hard_node_same14}%
}\qquad
\subfloat[Profil cu defect puternic în nodul 25]{%
  \includegraphics[width=0.5\textwidth]{\pics/c3_pics/emitter_node_same/emitter_hard_node25}%
  \label{fig:emitter_hard_node_same25}\qquad
}
\caption{Rezultate simulări defecte puternice}
\label{fig:ref_emitter_hard}
\end{figure}

Se observă de exemplu că pentru o valoare a emitter-ului de 100 caracteristica dinamică este deja modificată din cauza scurgerilor puternice din nod. 

Este relevantă împărțirea defectelor în mai multe clase de magnitudini pentru a putea valida un model de clasificare. Spre exemplu este normal să ne întrebăm dacă un model antrenat pe baza unui set date corespunzător unor magnitudini normale $C \in (0, 35)$ poate da rezultate consistente pentru un set de date cu magnitudini ale emitter-ului puternice $C \geqslant 35$. 

\section{Pre-procesarea datelor}
\label{sec:preproc}
În urma extragerii datelor din rețea este extrem de importantă etapa de prelucrare și pre-procesare a datelor. Domeniul de pre-procesare a datelor este unul extrem de vast și important în domeniul de învățare automată (engl. Machine Learning) și procesare de semnal. Pre-procesarea datelor este etapa în care datele de intrare pentru un algoritm sunt aduse la o formă optimă pentru desfășurarea procesului impus, de exemplu în domeniul clasificării este important ca algoritmii să primească date care să fie scalate într-un anumit domeniu, pentru a asigura convergența \cite{dataPreprocessing}, \cite{GIGO}. Alegerea metodei de pre-procesare este strâns legată de tipul de date disponibile și de starea acestora. În cazul rețelelor de apă, unde am ales caracteristica presiunii ca mărime de intrare pentru algoritm și ținând cont de răspunsul în timp al rețelei am considerat ca fiind necesare următoarele operații:

\begin{itemize}
\item eliminarea frontului comun și extragerea diferenței dintre semnalul nominal și cel măsurat în rețea
\item filtrarea semnalului obținut anterior
\end{itemize}

\section{Nomenclatura mărimilor alese}
\label{sec:nomenclatura}
Pentru a menține rigurozitatea și eleganța metodelor folosite este nevoie de o definire matematică pentru toate mărimile și metodele de filtrare folosite.

\subsection{Presiunea în regim dinamic}
Reprezintă o funcție de timp:
\begin{equation}
p_i : \mathbb{R} \longrightarrow \mathbb{R}^n, i \in V
\label{eq:press:func}
\end{equation}

unde $n$ reprezintă numărul de noduri al rețelei, iar $i$ reprezintă indexul nodului. 
Deoarece cazurile tratate în această lucrare reprezintă momente discrete de timp este important să definim presiunea măsurată în intervalele discrete în care este simulat procesul:
\begin{equation}
\mathbf{p}_i \in \mathbb{R}^{n \times p_{sim}}
\end{equation}

unde $p_{sim}$ reprezintă numărul de eșantioane pentru fiecare măsurătoare. Mergând mai departe în analiza simulării este de asemenea important să definim mărimea afectată de un defect în nodul $j$, de magnitudine $m$ și măsurată în nodul $i$:
\begin{equation}
\mathbf{p}^{j,m}_i \in \mathbb{R}^{n \times p_{sim}}
\end{equation}

Pentru cazul în care magnitudinea $m$ ia valori nule, atunci vom considera notația mărimii nominale:
\begin{equation}
\mathbf{p}^{j,0}_i = \mathbf{p}^{nom}_i, \forall j \in V
\end{equation}

Pentru valorile presiunii recoltate din rețea în nodul $i$ despre care nu se cunoaște nici o informație, vom considera notația
\begin{equation}
\widehat{\mathbf{p}}_i
\label{eq:measured_pressure}
\end{equation}

\subsection{Presiunea în regim static}
Considerând o plajă de momente de timp situate între indicii $rs_1 : rs_2$ unde se afla valorile de regim staționar ale procesului, putem defini o medie a regimului static în felul următor:
\begin{equation}
\overline{\mathbf{p}}_i^{j,m} = \frac{1}{rs_1 - rs_2 + 1} \sum_{k=rs_1}^{rs_2} \mathbf{p}_i^{j,m}[k] 
\label{eq:pmean}
\end{equation}

În aceeași manieră definim și media presiunii nominale în regim static:
\begin{equation}
\overline{\mathbf{p}}_i^{j,0} = \overline{\mathbf{p}_i}^{nom}, \forall j \in V
\label{eq:pmean_nom}
\end{equation}

Media presiunii măsurată în nodul $i$ și despre care nu se cunosc informații în legătură cu valoarea și poziția defectului, notație bazată pe relația \eqref{eq:measured_pressure}:
\begin{equation}
\overline{\widehat{\mathbf{p}}}_i = \frac{1}{rs_1 - rs_2 + 1} \sum_{k=rs_1}^{rs_2} \widehat{\mathbf{p}}_i[k] 
\label{eq:pmean_measured}
\end{equation}

\subsection{Reziduuri}
Așa cum a fost discutat în secțiunea \ref{sec:preproc}, pre-procesarea datelor are un rol important iar în cazul analizei și clasificării defectelor în rețelele cu apă, este nevoie să definim caracteristica prelucrată care va fi folosită mai apoi în procesul de izolare a defectelor. Reziduul absolut reprezintă diferența dintre valoarea măsurată în rețea și valoarea nominală, aici putem discerne două cazuri:

\begin{itemize}
    \item Reziduu imediat, definit ca un vector care conține diferențele presiunii măsurate de cea de referintă la fiecare moment de timp
    \item Reziduu mediat, definit ca diferența dintre mediile presiunilor pe intervalul regimului staționar
\end{itemize}

Reziduu imediat:
\begin{equation}
\mathbf{r}_i^{j,m} = \mathbf{p}_i^{j,m} - \mathbf{p}_i^{nom}
\label{eq:temp_residual}
\end{equation}
Reziduu mediat, calculat ca diferența dintre cele două valori mediate pe intervalul staționar al caracteristicii:
\begin{equation}
r_i^{j,m} = \overline{\mathbf{p}}_i^{j,m} - \overline{\mathbf{p}}_i^{nom}
\label{eq:absolute_residual}
\end{equation}

iar pentru valorile reziduului despre care nu se cunosc încă lucruri folosim notația din stilul anterior:
\begin{equation}
\widehat{r}_i = \overline{\widehat{\mathbf{p}}}_i - \overline{\mathbf{p}}_i^{nom}
\label{eq:measured_residual}
\end{equation}

Alte tipuri de reziduuri pre-procesate sunt relative:
\begin{equation}
rrelativ_i^{j,m} = \frac{r_i^{j,m}}{\overline{\mathbf{p}}_i^{nom}} 
\label{eq:relative_residual}
\end{equation}

Reziduurile normate:
\begin{equation}
rnorm_{i}^{j,m} =  \frac{r_i^{j,m}}{ \norm{r_{1:n}^{j,m}}} 
\label{eq:norm_residual}
\end{equation}

Reziduurile scalate:
\begin{equation}
rscal_{i}^{j,m} = \frac{r_i^{j,m} - \min r_{1:n}^{j,m}}{ \max r_{1:n}^{j,m} -  \min r_{1:n}^{j,m}}
\label{eq:scaled_residual}
\end{equation}


Ca semnificație notațiile prezentate în \ref{sec:nomenclatura} care conțin simbolul~ "$\widehat{}$" ~ fac referire la datele folosite pentru validarea modelului iar valorile unde se specifică nodul defectului și magnitudinea acestuia sunt considerate ca fiind date de antrenare și testare. Astfel în contextul definirii setului de dat pe care vom aplica algoritmii de clasificare va trebui să definim matricea:
\begin{equation}
\mathbf{R}<tip>^{j, m} \in \mathbb{R}^{n_{d} \times n}
\label{eq:residual_mat}
\end{equation}

Unde croșetele din formulă reprezintă un înlocuitor pentru metoda de reziduu folosită iar $n_{d}$ reprezintă numărul de defecte tratate în setul de date. De asemenea pentru fiecare linie a matricei \eqref{eq:residual_mat} putem defini perechea
\begin{equation}
\left( \mathbf{R}<tip>(d,:), y_{d} \right)
\label{eq:residual_mat_label}
\end{equation} 
Unde $ \mathbf{R}<tip>(d,:)$ reprezintă răspunsul rețelei prin reziduuri la defectul $d_i$. Iar $y_{d}$ reprezintă eticheta pentru acest set de date, anume, nodul în care a avut loc defectul.

\section{Calcul și prezentare reziduuri}
În continuare vom prezenta grafic reziduurile imediate normalizate pentru defecte în nodurile $V_d =\{11,17,27,29\}$ și măsurate în nodurile $V' = \{5,11,15,17,21,27 \}$

 
\begin{figure}[H]
\begin{tabular}{cc}
\subfloat[Reziduuri pentru defect în nodul 11, magnitudine 29]{%
  \includegraphics[width=0.4\textwidth]{\pics/c3_pics/residuals/time_res_emitter11_mag29}%
  \label{fig:residual_time_11}%
} &
\subfloat[Reziduuri pentru defect în nodul 17, magnitudine 29]{%
  \includegraphics[width=0.4\textwidth]{\pics/c3_pics/residuals/time_res_emitter17_mag29}%
  \label{fig:residual_time_17}%
} \\

\subfloat[Reziduuri pentru defect în nodul 21, magnitudine 29]{%
  \includegraphics[width=0.4\textwidth]{\pics/c3_pics/residuals/time_res_emitter21_mag29}%
  \label{fig:residual_time_21}%
}&

\subfloat[Reziduuri pentru defect în nodul 27, magnitudine 29]{%
  \includegraphics[width=0.4\textwidth]{\pics/c3_pics/residuals/time_res_emitter27_mag29}%
  \label{fig:residual_time_27}%
} 
\end{tabular}
\caption{Reziduuri imediate normalizate}
\label{fig:rez_time}
\end{figure}


Se poate observa că în figurile \ref{fig:rez_time} reziduul cel mai pronunțat ca funcție de timp se găsește în nodul în care se simulează și defectul - lucru natural și de așteptat. O caracteristică importantă a acestei rețele de apă este faptul că există o dependență între diferitele răspunsuri în timp ale caracteristicii de presiune, fapt care ne permite să exploatăm redundanțele din rețea și să prezicem cu o acuratețe relativ ridicată defectele.

Este necesar acum să prezentăm profilurile reziduurilor atemporale, care în final vor reprezenta caracteristicile de intrare pentru algoritmul de clasificare și selecție de senzori.

\begin{figure}[H]
\begin{tabular}{cc}
\subfloat[Reziduuri pentru defect în nodul 11, magnitudine 25]{%
  \includegraphics[width=0.5\textwidth]{\pics/c3_pics/residuals/atem_res_emitter11_mag25}%
  \label{fig:residual_atemp_11}%
} &
\subfloat[Reziduuri pentru defect în nodul 17, magnitudine 25]{%
  \includegraphics[width=0.5\textwidth]{\pics/c3_pics/residuals/atem_res_emitter17_mag25}%
  \label{fig:residual_atemp_17}%
} \\

\subfloat[Reziduuri pentru defect în nodul 21, magnitudine 25]{%
  \includegraphics[width=0.5\textwidth]{\pics/c3_pics/residuals/atem_res_emitter21_mag25}%
  \label{fig:residual_atemp_21}%
}&

\subfloat[Reziduuri pentru defect în nodul 27, magnitudine 25]{%
  \includegraphics[width=0.5\textwidth]{\pics/c3_pics/residuals/atem_res_emitter27_mag25}%
  \label{fig:residual_atemp_27}%
} 
\end{tabular}
\caption{Reziduuri atemporale rețea}
\label{fig:rez_atemp}
\end{figure}

Asemenea reziduurilor de la \ref{fig:rez_time} se poate observa că simularea unui \emph{emitter} într-un nod va determina un răspuns puternic în nodul respectiv și în vecinătatea nodului afectat.

\section{Metodă preliminară de selecție a senzorilor}
\label{sec:sensor_selection}
O metodă prin care se poate decide și evalua importanța senzorilor într-o rețea este prin construirea matricei de reziduuri \eqref{eq:residual_mat}, asupra căreia aplicăm o operație de scalare pe coloane, pentru a aduce valorile acesteia în intervalul $[0,1]$. Pentru experimentul în care simulăm în fiecare nod un emitter de 25 obținem grafic o matrice $\mathbf{R}scaled \in \mathbb{R}^{31 \times 31}$:

\begin{figure}[H]
\centering
\includegraphics[width=0.5\textwidth]{\pics/c3_pics/residuals/residual_scaled_matrix}
\caption{Matricea de reziduuri scalate, negru -- 1, alb -- 0}
\label{fig:rez_scaled_matrix_img}
\end{figure} 

După cum se observă în \ref{fig:rez_scaled_matrix_img} fiecare coloană a matricei $\mathbf{R}scaled$ reprezintă răspunsul prelucrat - scalat în intervalul $[0,1]$ - al unui nod, valorile care tind spre o culoare mai închisă de negru sunt răspunsuri mai pronunțate iar valorile care se apropie de alb reprezintă valori mai puțin evidențiate ale reziduului în nod (i.e., coloana de apă nu a scăzut atât de mult în nodul respectiv comparativ cu valoarea sa nominală). Analizând matricea se poate observa care noduri răspund mai bine la anumite răspunsuri, o caracteristică importantă a acestei matrice este că are o alură diagonală - semnificația elementelor diagonale fiind răspunsul nodurilor afectate de defecte aplicate în ele înseși. Dacă matricea de reziduuri ar fi avut o caracteristică diagonală perfectă - elementele nenule s-ar fi regăsit doar pe aceasta - problema de clasificare ar fi fost una trivială și ar fi implicat amplasarea unui senzor în fiecare nod - deși în același timp nepractic deoarece ar fi necesitat un număr foarte mare de senzori. Cazul real impune dependențe între presiunile, debitele din fiecare nod, dar și vitezele din fiecare conductă, astfel, fiecare defect separat va avea o "amprentă" unică - prin care se diferențiază de celelalte.. Scopul este găsirea unui subset al mulțimii de noduri care să poată identifica defectele care pot avea loc în tot setul de noduri, cu pierderi minime de acuratețe.


\subsection{Binarizarea matricei de reziduuri}

Pentru a putea reține un răspuns binar în cadrul matricei este important să considerăm o limită $l$ peste care răspunsul senzorului este luat în considerare sau nu. Astfel definim matricea binară $\mathbf{M}$ de aceeași dimensiune ca și $\mathbf{R}scaled$ dar peste care aplicăm o operație de binarizare:

\begin{equation}
    \mathbf{M}_{i,j} = \begin{cases}
            1 &\quad \text{ dacă } \mathbf{R}scaled_{i,j} \geq l \\
            0 &\quad \text{ dacă } \mathbf{R}scaled_{i,j}  < l
    \end{cases}
\label{eq:binary_matrix}
\end{equation}
În continuare vom defini notațiile matematice pentru mulțimile defectelor și a nodurilor.
Este important să considerăm astfel mulțimea răspunsurilor la defecte $F = \{\mathbf{f}_i | i \in V \}$ iar fiecare element $\mathbf{f}_i$ al mulțimii este definit ca un vector format din valori binare cu semnificația:

\begin{equation}
\mathbf{f}_i[k] = \begin{cases}
       1 &\quad\text{dacă nodul k răspunde la defectul i}\\
       0 &\quad\text{dacă nodul k nu răspunde la defectul i}
     \end{cases}
\label{eq:fault_signature_row}
\end{equation}

Echivalent putem defini mulțimea răspunsurilor nodurilor $N =\{\mathbf{n}_i| i \in V\}$, pentru fiecare element $\mathbf{n}_i$ avem:

\begin{equation}
\mathbf{n}_i[k] = \begin{cases}
       1 &\quad\text{dacă defectul k este detectat de nodul i}\\
       0 &\quad\text{dacă defectul k nu este detectat de nodul i}
     \end{cases}
\label{eq:fault_signature_column}
\end{equation}
Relația de la \ref{eq:fault_signature_row} definește o linie a matricei ilustrate în figura \ref{fig:rez_scaled_matrix_img} și definesc sensibilitatea nodurilor la un anumit defect.

În mod asemănător relația de la \ref{eq:fault_signature_column} definește o coloană matricei de reziduuri scalate și semnificația este sensibilitatea unui nod la toate defectele posibile.
\paragraph{Exemplificare} \mbox{} \\

Considerând o limită $l=0.65$ pentru matricea de răspunsuri scalate putem obține o matrice binară calculată după formula \eqref{eq:binary_matrix}. Forma matricei $\mathbf{M}$ este:
\begin{figure}[H]
\centering
\includegraphics[width=0.5\textwidth]{\pics/c3_pics/residuals/residual_binary_matrix}
\caption{Matricea de răspunsuri binarizate}
\label{fig:binary_matrix}
\end{figure}



\subsection{Selecția senzorilor}

Având definite elementele pentru interpretarea decizională a datelor preluate de la senzori, putem formula o metodă prin care să selectăm acei senzori care oferă informațiile cele mai importante. După cum se observă și în matricea reziduurilor \ref{fig:rez_scaled_matrix_img} există foarte multă redundanță în sistemul complex al rețelei de apă care întâmpină perturbații precum scurgeri. Ținând cont că de faptul că fiecare defect are un răspuns diferit, putem să decidem care este subsetul de senzori care reușește să ofere informații îndeajuns de relevante în legătură cu evenimentele din rețea, ca exemplu trivial putem afirma că un nod care răspunde la toate defectele oferă informații relevante pentru detecția de defecte dar nu și pentru izolarea defectelor.

Pentru problema selecției de senzori dorim să obținem o submulțime de noduri $V_s \subset V^j$ cu proprietatea că aceasta va avea un cardinal cât mai mic și defectele detectate de senzorii selectați sunt cât mai multe. Problema aceasta reprezintă de fapt o abordare a unei probleme NP-complete anume Minimum Set Cover - problema acoperirii minime a mulțimilor \cite{perelman2016sensor}.

\subsubsection{Problema MSC} 
Fiind dată o mulțime univers $U$ cu $|U| = n$ și o mulțime cu cardinal $m$ $S = \{S_i | S_i \subset U\}$ având proprietatea că  $\bigcup\limits_{i=1}^{m} S_{i} = U$ se dorește găsirea celei mai mici partiții a mulțimii $S$ a cărei reuniuni va fi egală cu $U$ \cite{CLRS}. Prin anumite relaxări ale problemei se poate ajunge la o soluție a cărei reuniune să fie cât mai apropiată de mulțimea univers $U$. 

În cazul rețelelor de apă mulțimea $U$ reprezintă de fapt mulțimea defectelor pentru care se dorește găsirea partiției cu cardinal minim a mulțimii de noduri care să fie sensibilă la toate defectele \cite{perelman2016sensor}.

Această problemă poate fi pusă sub forma unei probleme de optimizare matematică, având la dispoziție matricea binară $\mathbf{M}$ și un vector de selecție al senzorilor $\alpha$:
% \begin{equation*}
% \begin{align}
%     &\min_{\alpha \in \mathbb{R}^n} \sum_{j} \alpha_{j} \\
%     & s.l.: \sum_{j} M_{i,j}\alpha_{j} \geq 1,  i = \overline{1,n_{faults}}
% \end{align}
% \end{equation*}


\begin{subequations}
\begin{alignat}{2}
&\!\min_{\alpha \in \mathbb{R}^n}        &\qquad& \sum_{j} \alpha_{j} \label{eq:optProb}\\
&\text{s.l.:} &      & \sum_{j} M_{i,j}\alpha_{j} \geq 1 \label{eq:constr}
\end{alignat}
\label{eq:opt_problem}
\end{subequations}

 Minimizarea sumei vectorului $\alpha$ din \eqref{eq:optProb} se referă la selectarea unui număr cât mai mic de senzori pentru a fi plasați în rețea. Constrângerea atașată acestei probleme de programare liniară \eqref{eq:constr} are rolul de a impune ca pentru fiecare defect să existe cel puțin un senzor care să îl detecteze. Este important să amintim că parametrul $\alpha$ este de fapt o variabilă binară $\alpha = \{\alpha_i | \alpha_i \in \{0, 1\} \}$ ceea ce implică o dificultate sporită și solvere specializate \cite{gurobi} \cite{cplex}.
 
\subsubsection{Modalitatea de rezolvare a problemei MSC}
Având la dispoziție toate datele de intrare pentru problema MSC, am ales să folosesc toolbox-ul \textbf{YALMIP} - \textbf{MATLAB} pentru a afla nodurile cu importanța cea mai mare, rezolvând problema de optimizare \eqref{eq:opt_problem}. \textbf{YALMIP} este un toolbox specializat pentru probleme de optimizare și a fost dezvoltat așa fel încât programatorii să poată rezolva aceste ecuații într-un mod mult mai ușor - transpunând în cod ecuațiile de minimizare/maximizare și inegalitățile - fără să mai fie nevoie să aducă problema inițială la o formă intermediară \cite{YALMIP}.

Codul \textbf{MATLAB} care se rulează pentru a rezolva acestă problemă este:

\lstinputlisting[caption={Rezolvare problemă MSC},label={lst:MSC},firstline=55,lastline=66]{\code/sensor_selection.m}

Nodurile alese de această configurație a problemei sunt $V_s = \{11, 13, 15,20, 26, 27\}$ prezentate pe graful rețelei în figura următoare:

\begin{figure}[H]
\centering
\includegraphics[width=0.85\textwidth]{\pics/c3_pics/hanoi_network_sensors}
\caption{Rețeaua cu senzorii plasați}
\label{fig:binary_matrix}
\end{figure}

