\chapter{Clasificarea defectelor folosind metoda învățării de dicționare rare}
\label{chap:dictionary_learning}
\section{Aspecte teoretice}
Problema învățării dicționarelor se clasează în domeniul problemelor de învățare nesupervizată - setul de date nu trebuie să fie structurat pe clase - dar există și variațiuni ale acesteia prin care modifică problema de optimizare astfel încât să se poată învăța dicționare specializate pentru clasificare.

Având ca date de intrare un set de date $\mathbf{Y} \in \mathbb{R}^{n \times n_s}$, unde $n$ reprezintă dimensiunea semnalelor și $n_s$ reprezintă numărul de semnale folosite la antrenare, dorim să găsim acele matrice $\mathbf{D} \in \mathbb{R}^{n \times m}$ și $\mathbf{X} \in \mathbb{R}^{m \times n_s}$ astfel încât să rezolvăm problema de optimizare\cite[Capitol 2]{DL_book}:

\begin{subequations}
\begin{alignat}{2}
&\!\min_{\mathbf{D}, \mathbf{R}}        &\qquad& ||\mathbf{Y} - \mathbf{D} \mathbf{X} ||^{2}_{F} \label{eq:dl_opt}\\
&\text{s.l.:} &      & ||x_l||_0  \leq s, l = 1:N \label{eq:dl_sparsity}\\
& & & ||d_j|| = 1, j = 1:n \label{eq:dl_norm}
\end{alignat}
\label{eq:dl_opt_problem}
\end{subequations}
Unde:
\begin{itemize}
    \item $\mathbf{D}$ reprezintă matricea dicționarului pe baza căruia se va calcula reprezentarea, coloanele acesteia se numesc atomi
    \item $\mathbf{X}$ este reprezentarea rară a setului de date $\mathbf{Y}$
\end{itemize}

Constrângerea \eqref{eq:dl_sparsity} se referă la raritatea vectorului de reprezentare iar \eqref{eq:dl_norm} la normalizarea atomilor pentru dicționar.

Problema de găsire a dicționarului și a reprezentării semnalelor de antrenare $\mathbf{Y}$ conține neliniarități puternice din cauza condiției de sparsitate impuse. Cazul interesant și cel mai abordat în literatură îl reprezintă acela în care dicționarul este supracomplet \cite[Capitolul 1]{DL_book}, această proprietate poate aduce numeroase beneficii diferitelor procese de clasificare, anume:
\begin{itemize}
    \item stocarea matricelor sparse se face mult mai eficient decât cele pline
    \item din punct de vedere computațional există foarte multe multiplicări care nu se vor mai efectua
\end{itemize}

Modul în care am descris problema duce cu gândul la o metodă sofisticată de extragere a caracteristicilor din setul de date. Astfel dacă extragem o reprezentare sparsă $X = {x_i}$, putem să folosim vectorul rar $x_i$ ca exemplu de antrenare pentru alți algoritmi de antrenare sau clasificare.


TODO adauga clasificare etc




\section{Adaptarea la problema rețelelor de apă}

\section{Rezultate și metrici de clasificare}
