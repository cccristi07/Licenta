\chapter{Introducere}
\label{chap:intro}

\section{Motivația alegerii temei}
Transportul și distribuția apei reprezintă una dintre cele mai vechi preocupări inginerești de proporții, existând de mai mult de 4000 de ani. Civilizația minoică, localizată în insula Creta, este considerată a fi prima care a construit apeducte - structuri pentru transportul apei de la sursă către orașe - în 2500 î.Hr. 

Deși majoritatea popoarelor din antichitate care s-au ocupat cu construcția apeductelor întrebuințau aceste sisteme pentru irigația pământului - ocupațiile de bază de atunci fiind în strânsă legătură cu agricultura - romanii au văzut în sistemele de provizionare a apei și un potențial imens în dezvoltarea civilizației, astfel ei sunt ei care aduc cele mai mari contribuții inginerești, apeductele construite de aceștia impresionând și astăzi prin grandoarea și iscusunța cu care au fost construite.

Ținând cont de importanța apei în desfășurarea activităților cotidiene atât pentru oameni cât și pentru actorii importanți ai industriei, este o condiție sine-qua-non ca un oraș să aibă un sistem performant și rezistent la defecte pentru distribuția apei. În contextul actual al dezvoltării tehnologiei este natural să folosim tehnici moderne de monitorizare a diferiților parametrii din cadrul unei rețele pentru a putea face o analiză riguroasă și eficientă cu referire nu numai la mentenanță ci și la consumul global și local în ideea îmbunătățirii și reducerii pierderilor.


\section{Expunerea problemei}

În această lucrare se va aborda problematica identificării prezenței unui defect - \textit{Fault detection} și izolarea defectului \textit{Fault isolation} într-o regiune a rețelei.

O rețea de apă poate fi privită ca un graf neorientat $G = (V, E)$ unde $V$ este mulțimea nodurilor rețelei - acestea reprezentând o abstractizare asupra componentelor precum:
\begin{itemize}
\item rezervoare
\item tancuri de apă
\item puncte de distribuție
\end{itemize} 

$E$ este mulțimea muchiilor reprezentând de fapt țevile care fac legătura între noduri.

Mergând mai departe cu abstractizarea se pot considera rețele de apă active și rețele de apă pasive. Diferența între cele două făcându-se în baza pompelor de apă amplasate în zonele unde presiunea sau elevația vin în detrimentul distribuției apei.

Rețelele de apă care vor fi tratate în această lucrare fac parte din categoria pasivă, astfel putem diviza mulțimea nodurilor $V$ în $V^t$ și în $V^j$ reprezentând mulțimea nodurilor de tip tanc și mulțimea nodurilor joncțiune, cu proprietatea că $V = V^t \cup V^j$. Tancurile și rezervoarele dintr-o rețea de apă au proprietatea că nivelul de apă din acestea se va menține la un nivel oarecum staționar, astfel simulările din capitolele viitoare se vor axa pe nodurile simple de tip joncțiune, deci mulțimea de interes în acest caz va fi $V^j$ pentru care cunoaștem cardinalul.

Caracteristicile care se pot recolta dintr-o rețea de apă pot varia în funcție de elementul inspectat și de senzorii dispuși în rețea, astfel pentru fiecare nod $n_i \in V^j$ putem defini la fiecare moment de timp 
\begin{itemize}
\item presiunea $p_i(t)$ - măsurată în metri coloană de apă $mH2O$, mărime influențată puternic de presiunea interioară a nodului și de eventualele perturbații exterioare i.e. scurgeri de apă prin țevi
\item 'cererea' $d_i(t)$ - măsurată $L/s$, mărime ce caracterizează profilul de utilizare al utilizatorilor de-a lungul unei zile i.e. debitul de apă care ajunge la consumatori. Acest debit poate varia de-a lungul zilei, putem distinge de exemplu intervale de timp în care cererea este foarte mică și rețeaua intră în regim staționar
\end{itemize}

De asemenea pentru fiecare conductă a rețelei $e_{ij} \in E$ putem măsura viteza lichidului $v_{ij}(t)$.

Pentru a putea rezolva problema de \textit{Fault Detection and Isolation} este importantă găsirea unei modalități eficente de selecție și prelucrare a datelor de la rețea.Mai mult, punând în lumină aspectul ingineresc al problemei, trebuie găsită o submulțime $V_{opt} \subset V^j$ ai cărei elemente pot aduce informații necesare și suficiente pentru a detecta un defect într-o acoperire destul de mare a rețelei.

\section{Exemplul de lucru}
În următoarele capitole și în implementarea lucrării consider rețeaua din Hanoi iar pentru simularea scenariilor propuse voi folosi biblioteca și suita de funcții \textbf{EPANET} - Environmental Protection Agency NETwork - \cite{rossman2000epanet}

Rețeaua Hanoi constă într-o mulțime de noduri de tip joncțiune $V^j$ cu $|V^j| = 31$ și mulțimea $V^t$ cu $|V^t|=1$, ilustrată în figura de mai jos:
 
\begin{figure}[h]
\centering
\includegraphics[width=\textwidth]{pics/c1_pics/hanoi_network.pdf}
\caption{Graful rețelei de apă din Hanoi}
\end{figure}

După cum se poate observa în figura de mai sus au fost reprezentate două tipuri de node pasive, anume tancurile și joncțiunile. În cazul apariției unui defect în rețea, este important de luat în considerație modalitatea în care acesta va influența dinamica rețelei, spre exemplu este de la sine înțeles că dacă se cosideră un defect în nodul cu indicele 17 - i.e. în acest nod au apărut anumite scurgeri care afectează fluxul de apă către consumatori - nodurile în care se va observa o modificare puternică a caracteristicilor (presiune și debit) vor face parte din mulțimea nodurilor adiacente rețelei $S={V_{16}, V_{18}}$, deși pare o concluzie naturală, o modelare matematică riguroasă din care să se tragă această nu este o problemă foarte ușor de rezolvat, anumiți parametrii fiind extrem de greu de estimat chiar și în cazul în care se consideră un regim staționar.
