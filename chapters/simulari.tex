\chapter{Simulări și software folosit}
\label{chap:simulari}

\section{Dificultatea estimărilor parametrilor într-o rețea de apă}

Găsirea unui set de ecuații al cărei soluție să conducă la o estimare îndeajuns de bună pentru control este o condiție sine qua non pentru detecția unui defect și izolarea acestuia în cadrul nodurilor rețelei. Astfel după cum a fost expus în capitolul \ref{chap:intro} ecuațiile care guvernează relațiile intre viteza prin conducte și presiune dintr-un anumit punct sunt particularizări ale ecuațiilor Bernoulli-Euler sau Navier-Stokes. În cadrul unei rețele de apă a unui oraș, complexitatea rezolvării problemei crește semnificativ din varii motive precum:
\begin{itemize}
\item ansamblul de coduncte și noduri interconectate dă naștere unui sistem fizic greu de modelat matematic
\item parametrii care pot influența calitatea soluțiilor precum: tipul materialului conductei și al nodului, elevația fiecărui nod, rugozitatea fiecărei conducte și depunerile de pe aceasta
\item apariția unor factori exogeni care pot fi uneori greu de estimat - tiparul de utilizare al rețelei de către consumatori poate varia puternic
\item apariția defectelor precum scurgerile în proximitatea unui nod
\end{itemize}
