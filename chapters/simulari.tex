\chapter{Simulări și software folosit}
\label{chap:simulari}

\section{Dificultatea estimărilor parametrilor într-o rețea de apă}

Găsirea unui set de ecuații al cărei soluție să conducă la o estimare îndeajuns de bună pentru control este o condiție sine qua non pentru detecția unui defect și izolarea acestuia în cadrul nodurilor rețelei. Astfel după cum a fost expus în capitolul \ref{chap:intro} ecuațiile care guvernează relațiile intre viteza prin conducte și presiune dintr-un anumit punct sunt particularizări ale ecuațiilor Bernoulli-Euler sau Navier-Stokes. În cadrul unei rețele de apă a unui oraș, complexitatea rezolvării problemei crește semnificativ din varii motive precum:
\begin{itemize}
\item ansamblul de coduncte și noduri interconectate dă naștere unui sistem fizic greu de modelat matematic
\item parametrii care pot influența calitatea soluțiilor precum: tipul materialului conductei și al nodului, elevația fiecărui nod, rugozitatea fiecărei conducte și depunerile de pe aceasta
\item apariția unor factori exogeni care pot fi uneori greu de estimat - tiparul de utilizare al rețelei de către consumatori poate varia puternic
\item apariția defectelor precum scurgerile în proximitatea unui nod
\end{itemize}

Ținând cont de complexitatea problemei în regim dinamic pentru a putea obține o soluție de regim staționar a rețelei este necesar să ignorăm evenimentele imprevizibile precum apariția unei scurgeri sau variațiile bruște ale consumului.

Ecuațiile de regim staționar includ condiții de conservare fluxului de apă:

\begin{equation}
\label{Ecuația de conservare a rețelei de apă}
\sum\limits_{j=1}^{n} \mathbf B_{ij}\mathbf q_j=\mathbf d_i
\end{equation}

Unde $q_i$ reprezintă debitul prin fiecare conductă iar \textbf{B} reprezintă matricea de adiacență a rețelei la echilibru, definită astfel
\begin{equation}
\textbf{B}_{ij} = 
     \begin{cases}
       1, & \text{conducta j intră în nodul i}\\
       0, & \text{conducta j nu este conectată la nodul i} \\
       -1, & \text{conducta j iese din nodul i}\\ 
     \end{cases}
\end{equation}

Partea de estimare a diferenței de presiuni (în engl. "Head-Flow differential") între două noduri interconectate se face utilizând formula Hazen-Williams \cite{sanz2016demand}:
\begin{equation}
\label{debit_presiune}
\mathbf h_i-\mathbf h_j=\frac{10.67\cdot L_\ell}{C_\ell^{1.852}\cdot D_\ell^{4.87}}\cdot \mathbf q_\ell\cdot |\mathbf q_\ell|^{0.852}
\end{equation}

unde:
\begin{itemize}
\label{Hazen-Williams}
\item $\textbf{h}$ reprezintă presiunea - măsurată de obicei în metru coloană de apă
\item $C_l$  reprezintă coeficientul de rugozitate al conductei
\item $D_l$ reprezintă diametrul conductei
\item $L_l$ reprezintă lungimea conductei
\item $q_l$ reprezintă debitul
\end{itemize}

Din ecuația empirică \eqref{Hazen-Williams} termenul $R_{ij}=\frac{10.67\cdot L_\ell}{C_\ell^{1.852}\cdot D_\ell^{4.87}}$ reprezintă rezistența conductei $ij$ iar dual, putem obține conductivitatea conductei $G_{ij} = \frac{1}{R_{ij}}$

Având la dispoziție \eqref{Hazen-Williams} și \eqref{debit_presiune} putem exprima dependența debit presiune în regim staționar sub o formă matriceală compactă și cu o structură neliniară:

\begin{equation}
\mathbf B\mathbf G\left[\left(-\mathbf B^\top \mathbf h+\mathbf B_f^\top \mathbf h_f\right)\times \left|-\mathbf B^\top \mathbf h+\mathbf B_f^\top \mathbf h_f\right|^{-0.46}\right]=\mathbf d
\end{equation}

unde s-au luat în calcul și nodurile care au variații de presiune foarte mici - spre exemplu nodurile de tip tanc și  nodurile de tip rezervor - termenul $\mathbf B_f^\top \mathbf h_f$ reprezintă contribuția acestor noduri la starea de echilibru a rețelei.

Din cauza dificultății rezolvării unei ecuații matriceale neliniare, software-ul specializat trebuie să folosească diferite metode de optimizare ("Solver") pentru a putea obține o diferență cât mai mică între cazul estimat și rezultatul real al ecuației. Este important de reținut faptul că rezolvarea problemelor de programare neliniară cu constrângeri poate generea de fapt o problemă NP-completă, sau în unele cazuri chiar NP-dură \cite{karp1975computational}.

\section{Simulări folosind biblioteca EPANET}